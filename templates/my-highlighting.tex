\providecommand{\SpecialStringTok}[1]{\textcolor[rgb]{0.31,0.60,0.02}{#1}}}
\providecommand{\CommentTok}[1]{\textcolor[rgb]{0.56,0.35,0.01}{\textit{#1}}}```

**Modified Code in `my-highlighting.tex` (`pdfx`-Compatible):**

```latex
% ==========================================================
% PDF/X COMPATIBLE SYNTAX HIGHLIGHTING DEFINITIONS
% ==========================================================

% 1. Define all code colors with unique names first.
% This is the crucial step for pdfx compliance.
\definecolor{pandocSpecialString}{rgb}{0.31,0.60,0.02}
\definecolor{pandocComment}{rgb}{0.56,0.35,0.01}
\definecolor{pandocKeyword}{rgb}{0.13,0.29,0.53}
\definecolor{pandocDataType}{rgb}{0.13,0.29,0.53}
\definecolor{pandocDecVal}{rgb}{0.00,0.59,0.00}
\definecolor{pandocOperator}{rgb}{0.81,0.36,0.00}
\definecolor{pandocError}{rgb}{0.64,0.00,0.00}
\definecolor{pandocConstant}{rgb}{0.90,0.00,0.00}
% ... add a \definecolor for EVERY color used in your highlighting theme ...

% 2. Now, redefine the Pandoc commands to use these named colors.
\providecommand{\AlertTok}[1]{\textcolor{pandocError}{#1}}
\providecommand{\AnnotationTok}[1]{\textcolor{pandocComment}{\textbf{\textit{#1}}}}
\providecommand{\AttributeTok}[1]{\textcolor{pandocKeyword}{#1}}
\providecommand{\BaseNTok}[1]{\textcolor{pandocDecVal}{#1}}
\providecommand{\BuiltInTok}[1]{#1} % Usually black
\providecommand{\CharTok}[1]{\textcolor{pandocSpecialString}{#1}}
\providecommand{\CommentTok}[1]{\textcolor{pandocComment}{\textit{#1}}}
\providecommand{\CommentVarTok}[1]{\textcolor{pandocComment}{\textbf{\textit{#1}}}}
\providecommand{\ConstantTok}[1]{\textcolor{pandocConstant}{#1}}
\providecommand{\ControlFlowTok}[1]{\textcolor{pandocKeyword}{\textbf{#1}}}
\providecommand{\DataTypeTok}[1]{\textcolor{pandocDataType}{#1}}
\providecommand{\DecValTok}[1]{\textcolor{pandocDecVal}{#1}}
\providecommand{\DocumentationTok}[1]{\textcolor{pandocComment}{\textbf{\textit{#1}}}}
\providecommand{\ErrorTok}[1]{\textcolor{pandocError}{\textbf{#1}}}
\providecommand{\ExtensionTok}[1]{#1}
\providecommand{\FloatTok}[1]{\textcolor{pandocDecVal}{#1}}
\providecommand{\FunctionTok}[1]{\textcolor{pandocComment}{#1}}
\providecommand{\ImportTok}[1]{#1}
\providecommand{\InformationTok}[1]{\textcolor{pandocComment}{\textbf{\textit{#1}}}}
\providecommand{\KeywordTok}[1]{\textcolor{pandocKeyword}{\textbf{#1}}}
\providecommand{\NormalTok}[1]{#1}
\providecommand{\OperatorTok}[1]{\textcolor{pandocOperator}{\textbf{#1}}}
\providecommand{\OtherTok}[1]{\textcolor{pandocComment}{#1}}
\providecommand{\PreprocessorTok}[1]{\textcolor{pandocComment}{\textit{#1}}}
\providecommand{\RegionMarkerTok}[1]{#1}
\providecommand{\SpecialCharTok}[1]{\textcolor{pandocOperator}{\textbf{#1}}}
\providecommand{\SpecialStringTok}[1]{\textcolor{pandocSpecialString}{#1}}
\providecommand{\StringTok}[1]{\textcolor{pandocSpecialString}{#1}}
\providecommand{\VariableTok}[1]{\textcolor{pandocDecVal}{#1}}
\providecommand{\VerbatimStringTok}[1]{\textcolor{pandocSpecialString}{#1}}
\providecommand{\WarningTok}[1]{\textcolor{pandocComment}{\textbf{\textit{#1}}}}

% Note: The Shaded environment for the background also needs to be defined here.
\usepackage{tcolorbox}
\tcbuselibrary{breakable}
\definecolor{codebg}{HTML}{F5F5F5} % This is a light gray
\newenvironment{Shaded}{\begin{tcolorbox}[breakable, colback=codebg, colframe=codebg, boxrule=0pt, sharp corners, left=2mm, right=2mm, top=2mm, bottom=2mm]}{\end{tcolorbox}}
\newenvironment{Highlighting}{}{}
